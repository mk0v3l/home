\documentclass {article}
\usepackage{minted, xcolor, graphicx, caption, geometry}
\geometry{
  left=2.5cm,
  right=2.5cm,
  top=2.5cm,
  bottom=3cm
}
\captionsetup{font=footnotesize}


\begin {document}

\begin{figure}[t]
    \centering{\includegraphics[scale=0.5]{~/templates/latex/images/ulbLogo.png}}
	\label{fig:ulbLogo}
\end{figure}
\author {Mickael Kovel 000396950}
\date {2 Mai 2023}
\title {Template}
\maketitle
\newpage
\tableofcontents
\newpage


\section {Introduction}

\section {Configuration}

\begin{minted}{latex}
\documentclass {article}
\usepackage{minted, xcolor, graphicx, caption, geometry}
\geometry{
  left=2.5cm,
  right=2.5cm,
  top=2.5cm,
  bottom=3cm
}
\captionsetup{font=footnotesize}


\begin {document}
\end{minted}
\section {Titre}

\begin{minted}{latex}
\begin{figure}[t]
    \centering{\includegraphics[scale=0.5]{~/templates/latex/images/ulbLogo.png}}
	\label{fig:ulbLogo}
\end{figure}
\author {Mickael Kovel 000396950}
\date {2 Mai 2023}
\title {Template}
\maketitle
\newpage
\tableofcontents
\newpage
\end{minted}


\newpage
\section {Figures}
\subsection {Simple}
\begin{figure}[h]
	\centering{\includegraphics[scale=0.5]{~/templates/latex/images/ulbLogo.png}}
	\caption{Logo de l'ULB}
	\label{fig:ulbLogo}
\end{figure}

\begin{minted}{latex}

\begin{figure}[h]
	\centering{\includegraphics[scale=0.5]{~/templates/latex/images/ulbLogo.png}}
	\caption{Logo de l'ULB}
	\label{fig:ulbLogo}
\end{figure}


\end{minted}

\newpage
\subsection {Côte à côte}

\begin{figure}[h]
    \begin{minipage}[t]{0.50\textwidth}
	\centering
	\includegraphics[width=\textwidth]{~/templates/latex/images/ulbLogo.png}
	\caption{Logo de l'ULB}
	\label{fig:model1}				
    \end{minipage}			
    \begin{minipage}[t]{0.50\textwidth}				
	\centering
	\includegraphics[width=\textwidth]{~/templates/latex/images/ulbLogo.png}
	\caption{Logo de l'ULB}
	\label{fig:model2}
    \end{minipage}
\end{figure}
\begin{verbatim}

\begin{figure}[h]
    \begin{minipage}[t]{0.50\textwidth}
	\centering
	\includegraphics[width=\textwidth]{~/templates/latex/images/ulbLogo.png}
	\caption{Logo de l'ULB}
	\label{fig:model1}				
    \end{minipage}			
    \begin{minipage}[t]{0.50\textwidth}				
	\centering
	\includegraphics[width=\textwidth]{~/templates/latex/images/ulbLogo.png}
	\caption{Logo de l'ULB}
	\label{fig:model2}
    \end{minipage}
\end{figure}

\end{verbatim}


\section {Références}
Comment référencer une figure : \ref{fig:ulbLogo}
\begin{verbatim}
Comment référencer une figure : \ref{fig:ulbLogo}
\end{verbatim}

\newpage
\section {Code}
\subsection {Code C++}
\begin{minted}{cpp}
#include <iostream>
using namespace std;
int main() {
	cout << "Hello, World!";
	return 0;
}
\end{minted}

\begin{verbatim}
\begin{minted}{cpp}
#include <iostream>
using namespace std;
int main() {
	cout << "Hello, World!";
	return 0;
}
\end{minted}
\end{verbatim}

\subsection {Code Python}

\begin{minted}{python}
def main():
	print("Hello, World!")
if __name__ == "__main__":
    main()
\end{minted}

\begin{verbatim}
\begin{minted}{python}
def main():
	print("Hello, World!")
if __name__ == "__main__":
	main()
\end{minted}
\end{verbatim}

\subsection {Code Java}
\begin{minted}{java}
public class HelloWorld {
	public static void main(String[] args) {
		System.out.println("Hello, World!");
	}
}
\end{minted}

\begin{verbatim}
\begin{minted}{java}
public class HelloWorld {
	public static void main(String[] args) {
		System.out.println("Hello, World!");
	}
}
\end{minted}
\end{verbatim}

\subsection {Code Bash}
\begin{minted}{bash}
#!/bin/bash
echo "Hello, World!"
\end{minted}

\begin{verbatim}
\begin{minted}{bash}
#!/bin/bash
echo "Hello, World!"
\end{minted}
\end{verbatim}



\subsection {Code Matlab}

\begin{minted}{matlab}
D = 20;gamma = ones(1,nn+1)/D; 
y1 = 100.*ones(1,10);y2 = 75.*ones(1,10);y3 = 50.*ones(1,10);y4 = 25.*ones(1,10);y5 = 0.*ones(1,10);
gamma(1:50) = [y1 y2 y3 y4 y5].*gamma(1:50);
\end{minted}

\begin{verbatim}
\begin{minted}{matlab}
D = 20;gamma = ones(1,nn+1)/D; 
y1 = 100.*ones(1,10);y2 = 75.*ones(1,10);y3 = 50.*ones(1,10);y4 = 25.*ones(1,10);y5 = 0.*ones(1,10);
gamma(1:50) = [y1 y2 y3 y4 y5].*gamma(1:50);
\end{minted}
\end{verbatim}

\end {document}
